% Notes
% - Local snippets in ./snips:
%   - (\)chap(tertoc)
%   - (\)rec(ipe)
%   - (\)ere(cipe)
% - Custom Commands:
%   - \chapterrec{title}       - usual chapter meant to have a fore-text
%   - \chaptertoc{title}       - chapter with table of contents
%   - \recipe{title}           - custom section command
%   - \erecipe{title}{source}  - external recipe, adds ToC entry
%   - \itemNL                  - item without label
%   - \namepageref[<s>]{label} - "<name><s><pg>" sep defaults to " on page "
%   - \degF, \about            - cooking shorthand
% - Custom Environments:
%   - ingredients, timeline, directions, directions1col [<alt title>]
%   - IT [<col ratio>]  (ingredients/timeline in 2 column)
%   - twocols
%   - ingredientslist, timelinelist, directionslist, recipelist [enumitem options]
% - How to use the index:
%   - \index{simple entry}
%   - \index{category!entry}          - category,\\  entry
%   - \index{entry!|see{other entry}} - entry ... /see/ other entry
%
% TODO:
% - 

\documentclass[oneside]{book}  % {{{1
\usepackage{lmodern, microtype}              % margins, fonts, etc
\usepackage[T1]{fontenc}                     % ""
\usepackage{calc, etoolbox, nameref, xspace} % utilities
\usepackage{booktabs, enumitem}              % typesetting
\usepackage{datetime2}                       % use YYYY-MM-DD for \today
\usepackage{hyperref}                        % pdf links

% spacing
\usepackage[margin=1in]{geometry}
\setlength{\parskip}{0.5em}

\usepackage{chappg}  % page numbering by chapter: <chapter #>-<page #>

% headers and footers
\usepackage{fancyhdr}
\fancypagestyle{plain}{  % redefine the plain pagestyle
  % Clear all headers and footers, show the date at the top right, and show
  % chapter name, recipe name, and page number on the bottom of the page.
  \fancyhf{}
  \fancyhead[R]{\scriptsize\today}
  \fancyfoot[L]{\leftmark\thisrecipe} \fancyfoot[R]{\thepage}
  % double-sided
  % \fancyhead[RE]{\scriptsize\today}
  % \fancyfoot[LE]{\leftmark\thisrecipe} \fancyfoot[RE]{\thepage}
  % \fancyhead[LO]{\scriptsize\today}
  % \fancyfoot[LO]{\thepage}             \fancyfoot[RO]{\leftmark\thisrecipe}
}
\renewcommand{\headrulewidth}{0em}               % remove header rule
\renewcommand{\chaptermark}[1]{\markboth{#1}{}}  % don't show chapter number
\pagestyle{plain}

% chapters, sections, etc, and ToC (NB: displaying numbers is disabled)
\usepackage{titlesec, titletoc}
\titleformat{\chapter}[display]{\normalfont\bf\Huge}{}{0em}{}
\titlespacing{\chapter}{0em}{-6em}{0em}
\titleformat{\section}{\normalfont\bf\Large}{}{0em}{}[\titlerule]
\pretocmd{\section}{\clearpage}{}{}  % start sections on a new page
\titleformat{\subsection}{\normalfont\bf\large}{}{0em}{}
\titlespacing{\subsection}{0em}{\parskip}{0em}
\titlecontents{chapter}[0em]{\addvspace{1em}\bf}{}{}{~\titlerule*[0.5em]{·}~\contentspage}
\titlecontents{section}[1.5em]{}{}{}{~\titlerule*[0.5em]{·}~\contentspage}
% chapter ToCs: style the sections the same as the chapters (styled above)
\titlecontents{chapter_toc_section}[0em]{\addvspace{1em}\bf}{}{}{~\titlerule*[0.5em]{·}~\contentspage}
% ugly hack: footer normal case, consistent spacing with chapter titles
\renewcommand{\contentsname}{\vspace{1.25em}\NoCaseChange{Contents}}
\setcounter{tocdepth}{1}  % don't show subsections and below in ToC

% custom commands for chapters of recipes and recipes
\def\thisrecipe{}  % initialization
\newcommand{\chapterrec}[1]{  % recipe chapter
  \newpage \def\thisrecipe{} \chapter{#1} \vspace{1.1em}
}
\newcommand{\chaptertoc}[1]{  % recipe chapter with table of contents
  \chapterrec{#1} \vspace{-1.1em}  % make spacing consistent
  \startcontents[chapters] \printcontents[chapters]{chapter_toc_}{1}[1]{}
}
\newcommand{\recipe}[1]{\section{#1}\def\thisrecipe{: #1}} % recipe section
\usepackage{xcolor}  % for \colorbox below
\NewDocumentCommand{\ExternalToCEntry}{mm}{\addtocontents{toc}{\protecting{%
  \contentsline{section}
  {\numberline{}#1}
  {\hspace{-\linewidth}\colorbox{white}{#2}\hspace{-0.3em}}  % \colorbox and negative \hspace are hacks to prevent overfull hboxes and the \dotfill from going through the page number
  {}
} } }
\NewDocumentCommand{\erecipe}{mm}{\ExternalToCEntry{#1}{#2}}  % external recipe

% index
\usepackage{imakeidx}
\makeindex[
  intoc,
  % ugly hack: footer normal case, consistent spacing with chapter titles
  title={\texorpdfstring{\vspace{1.25em}\NoCaseChange{Index}\vspace{0.4em}}{Index}},
  options=-s my-index,
]

% lists
\renewcommand*{\arraystretch}{1.25}
\setlist[enumerate]{itemsep=0.5em}
\setlist[itemize]{itemsep=0.5em}
\setlist[description]{itemsep=0.5em}
\newlist{kitchennotes}{description}{1}
\setlist[kitchennotes]{style=nextline, font=\bfseries, leftmargin=\parindent}
\newlist{kitchensubnoteslist}{itemize}{1}
\setlist[kitchensubnoteslist]{
  font=\bfseries, label=\textbullet, leftmargin=1em, itemsep=0em, topsep=0.5em
}
\NewDocumentEnvironment{kitchensubnotes}{o}
  {
    \phantom{.} \vspace{-2em}
    \IfNoValueTF{#1}{\begin{kitchensubnoteslist}}{\begin{kitchensubnoteslist}[#1]}
  }
  {
    \end{kitchensubnoteslist}
  }
\newlist{ingredientslist}{enumerate}{1}
\newlist{timelinelist}{enumerate}{1}
\newlist{directionslist}{enumerate}{1}
\newlist{recipelist}{enumerate}{1}
\SetEnumitemKey{recipelistStyle}{
  label=\arabic*., font=\bfseries, wide=\parindent, parsep=\parskip,
  topsep=\parskip, labelindent=\parindent,
}
\SetEnumitemKey{ITStyle}{itemsep=0em, label={}, leftmargin=*}
\setlist[ingredientslist]{recipelistStyle, ITStyle}
\setlist[timelinelist]{recipelistStyle, ITStyle}
\setlist[directionslist]{recipelistStyle, itemsep=\parskip}
\setlist[recipelist]{recipelistStyle}
\newcommand{\itemNL}{\item[] \hspace{-\labelsep}}  % item with no label

% custom commands
\renewcommand{\epsilon}{\varepsilon}
\renewcommand{\phi}{\varphi}
\NewDocumentCommand{\todo}{o}{\textbf{ToDo\IfNoValueF{#1}{: #1}}}
\NewDocumentCommand{\namepageref}{O{ on page }m}{\nameref{#2}#1\pageref{#2}}
\newcommand{\degF}{\textdegree F\xspace}
\newcommand{\about}{$\sim$}

% custom environments
\usepackage{multicol, paracol}  % multicolumn environments
\setlength{\multicolsep}{\parskip}
\NewDocumentEnvironment{twocols}{}  % shortcut for \begin{multicols}{2} ...
  {\begin{multicols}{2}}{\end{multicols}}
\NewDocumentEnvironment{IT}{O{0.844}}  %  [<col ratio>], ingredients/timeline
  {\columnratio{#1} \begin{paracol}{2}}{\end{paracol}}
\NewDocumentEnvironment{ingredients}{O{Ingredients}}
  {\subsection{#1} \begin{ingredientslist}}
  {\end{ingredientslist}}
\NewDocumentEnvironment{timeline}{O{Timeline}}
  {\subsection{#1} \begin{timelinelist}}
  {\end{timelinelist}}
\NewDocumentEnvironment{directions}{O{Directions}}
  {
    \vspace{-\parskip} \subsection{#1}
    \begin{twocols} \begin{directionslist}
  } {
    \end{directionslist} \end{twocols}
  }
\NewDocumentEnvironment{directions1col}{O{Directions}}
  {\subsection{#1} \begin{directionslist}}
  {\end{directionslist}}

% sources
\NewDocumentCommand{\esource}{mo}{#1\IfNoValueF{#2}{ p#2}\xspace}
\NewDocumentCommand{\CICookbook}{o}{\esource{CI Cookbook}[#1]}
\NewDocumentCommand{\CIMeat}{o}{\esource{CI Meat Book}[#1]}
\NewDocumentCommand{\CIBaking}{o}{\esource{CI Baking Book}[#1]}
\NewDocumentCommand{\CIScience}{o}{\esource{CI Science}[#1]}
\NewDocumentCommand{\InvIndianCooking}{o}{\esource{Inv.\ to Indian Cooking}[#1]}
\NewDocumentCommand{\ATKThirty}{o}{\esource{ATK 30-min}[#1]}
%----------------------------------------------------------------------------}}}1

\begin{document}
\frontmatter \pagenumbering[C]{bychapter} \tableofcontents

\mainmatter

\chapterrec{Notes} \label{chap:notes} % {{{1
\begin{kitchennotes}[font=\MakeUppercase]
  \item[Baking Sheets]
    Use LEGO blocks to build spacers for stacking cookie sheets.

  \item[Bulk Mixed Drinks]
    Pre-mix the drink and put it in the freezer. Make a pitcher of ice water and
    put it in the refrigerator. Work out correct ratio of drink to water (2:1 is
    a good starting point). Combine at time of serving.

  \item[Non-stick Pans]
    Use vegetable oil spray (e.g.\ PAM) on a steel pan. The emulsifiers make the
    pan essentially non-stick.

  \item[Potatoes] \begin{kitchensubnotes}
    \item Boursin cheese can be used to make mashed potatoes. Add a package, use
      less butter, and use milk instead of cream.
    \item Cut potatoes can be stored in water for up to one day.
  \end{kitchensubnotes}
\end{kitchennotes}
%----------------------------------------------------------------------------}}}1

\chapterrec{Sous Vide} \label{chap:sous_vide} % {{{1
\index{sous vide}
\begin{kitchennotes}
  \item[Chicken, Poached]
    \index{Poached Chicken} \index{sous vide!Poached Chicken} \index{chicken!Poached Chicken}
    Cook at 150\degF for at least 2 hours.

  \item[Eggs, Soft Boiled]
    \index{Soft Boiled Eggs} \index{sous vide!Yogurt} \index{eggs!Soft Boiled}
    Heat water to 190\degF and add eggs using spoon. Cook for 9--10 minutes,
    then place in cold water bath.

  \item[Yogurt]
    \index{Yogurt} \index{sous vide!Yogurt}
    Pasteurize the milk for 1 hour at 185\degF. Cool in a water bath to
    115\degF, about 15 minutes. Add yogurt starter and cultivate at 115\degF
    for 4 hours. Strain with colander and coffee filter for 3 hrs.
\end{kitchennotes}
%----------------------------------------------------------------------------}}}1

\chaptertoc{Appetizers} \label{chap:appetizers}
\recipe{Beef Tartare} \label{recipe:beef_tartare} % {{{1
\index{beef!Beef Tartare}

\begin{IT}
  \begin{ingredients}
    \item[1/2 lb] beef sirloin or chuck
    \item[2 Tbsp] olive oil
    \item[1 Tbsp] whole grain mustard
    \item[1 tsp] Dijon mustard
    \item[1] shallot, diced
    \item[1 Tbsp] capers, drained, rinsed, and diced
    \item[1 tsp] Worcestershire sauce
    \item[1 tsp] red wine vinegar
    \item[1] egg yolk
    \itemNL salt and pepper, to taste
  \end{ingredients}

  \switchcolumn

  \begin{timeline}
    \item[Prep:]  00:20
    \item[Cook:]  00:00
    \item[Total:] 00:20
  \end{timeline}
\end{IT}

\begin{directions}
  \item Place the meat grinder, serving bowl, and beef in freezer for 15
    minutes.
  \item Combine all ingredients except the beef. Grind beef and combine. Serve
    in frozen serving bowl.
\end{directions}
%---------------------------------------------------------------------------}}}1

\chaptertoc{Dinners (with leftovers)} \label{chap:dinners_with_leftovers}
\erecipe{Bolognese Sauce}{\CICookbook[193]} % {{{1
\index{Bolognese Sauce|see{\CICookbook[193]}}
\index{pasta!Bolognese Sauce|see{\CICookbook[193]}}
%---------------------------------------------------------------------------}}}1
\erecipe{Brisket}{\CIMeat[127]} % {{{1
\index{Brisket|see{\CIMeat[127]}}
\index{beef!Brisket|see{\CIMeat[127]}}
%---------------------------------------------------------------------------}}}1
\recipe{Chili} \label{recipe:chili} % {{{1
\index{Chili}
\index{beef!Chili}

\begin{IT}
  \begin{ingredients}
    \item[2] onion, diced
    \itemNL spicy peppers, diced
    \item[2 lbs] ground beef
    \item[2 Tbsp] cumin
    \item[1 Tbsp] chili powder
    \item[4 cans] green chilies
    \item[4 cans] kidney beans
    \item[2 can] black beans
    \item[4 cans] diced tomatoes
    \item[1 can] tomato sauce
    \item[1 bag] frozen corn
    \itemNL salt and pepper, to taste
  \end{ingredients}

  \switchcolumn

  \begin{timeline}
    \item[Prep:]  00:30
    \item[Cook:]  3:30
    \item[Total:] 4:00
  \end{timeline}
\end{IT}

\begin{directions}
  \item Saute the onions. Once onion is softened, cook diced chilies and spices
    for \about 1 minute. Brown the ground beef.

  \item Combine all remaining ingredients except corn and black beans in large
    pot. Bring to boil, then reduce heat to bare simmer. Cook for 2--3 hours.

  \columnbreak

  \item In final hour of cooking, add black beans and corn.
\end{directions}
%---------------------------------------------------------------------------}}}1
\erecipe{Meatloaf}{\CICookbook[391]} % {{{1
\index{Meatloaf|see{\CICookbook[391]}}
%---------------------------------------------------------------------------}}}1
\erecipe{Pork Roast}{\textit{see} \namepageref[ ]{notes:dinners_leftovers}}
\erecipe{Pulled Pork}{\CIMeat[244]} % {{{1
\index{Pulled Pork|see{\CIMeat[244]}}
\index{pork!Pulled Pork|see{\CIMeat[244]}}
%---------------------------------------------------------------------------}}}1
\recipe{Punjabi Meat Curry} \label{recipe:punjabi_meat_curry} % {{{1
\index{Punjabi Meat Curry}
\index{curry!Punjabi Meat Curry}
\index{beef!Punjabi Meat Curry}
\index{goat!Punjabi Meat Curry}

\begin{IT}
  \begin{ingredients}
    \item[5 lbs] red meat (e.g.\ goat or beef)
    \item[1/2 cup] olive oil
    \item[1 tsp] cumin seeds
    \item[2] green cardamom seeds, crushed
    \item[1] bay leaf
    \item[3] yellow onions, minced finely
    \item[3 Tbsp] ginger, chopped
    \item[6] cloves garlic, minced
    \item[4] chili peppers (e.g.\ Serrano)
    \item[3] Carolina reaper peppers
    \item[2 tsp] paprika powder
    \item[1 Tbsp] turmeric powder
    \item[1 Tbsp] chili powder
    \item[1 Tbsp] coriander powder
    \item[3 cans] tomato sauce (1 large can, 1 smaller)
    \item[8--10 cups] chicken broth or water
    \item[2 tsp] garam masala
    \itemNL cilantro
  \end{ingredients}

  \switchcolumn

  \begin{timeline}
    \item[Prep:]  00:30
    \item[Cook:]  3:00
    \item[Total:] 3:30
  \end{timeline}
\end{IT}

\begin{directions}
  \item Heat oil in a pot over medium high heat until shimmering. Add the cumin
    seeds, cardamom and bay leaf and saute for 1 minute, stirring to prevent the
    spices from burning. Add the onion and saute until the onion begins to turn
    light brown. Add the ginger, garlic and chilies, stirring to prevent
    burning, and cook 2 to 3 minutes. Add the meat and cook until all sides are
    browned, 10 to 15 minutes.

  \item Add the paprika, turmeric, red chili powder, and coriander powder. Stir
    well and then add the tomato sauce. Stir and add the water. Increase the
    heat to high and bring the curry to a boil. Reduce the heat to medium-low
    and continue to simmer until the meat is tender.

  \columnbreak

  \item Once the meat is tender and the liquid has reduced, add garam masala.
    Garnish with chopped cilantro leaves and serve.
\end{directions}
%---------------------------------------------------------------------------}}}1
\erecipe{Red Beans and Rice}{\CICookbook[239]} % {{{1
\index{Red Beans and Rice|see{\CICookbook[239]}}
\index{beans!Red Beans and Rice|see{\CICookbook[239]}}
%---------------------------------------------------------------------------}}}1
\erecipe{Roast Leg of Lamb}{\textit{see} \namepageref[ ]{notes:dinners_leftovers}}
\erecipe{Smothered Pork Chops}{\CIMeat[203]} % {{{1
\index{Smothered Pork Chops|see{\CIMeat[203]}}
\index{pork!Smothered Pork Chops|see{\CIMeat[203]}}
%---------------------------------------------------------------------------}}}1
\recipe{Sopa Seca de Fideos} \label{recipe:sopa_seca_de_fideos} % {{{1
\index{Sopa Seca de Fideos}

\begin{IT}
  \begin{ingredients}
    \item[5 lbs] tomatoes
    \item[8] garlic cloves, unpeeled
    \item[12 oz] chorizo sausage
    \item[20 oz] fideo noodles
    \item[2 cans] chilies in adobo
    \item[2] Carolina reaper peppers
    \item[2 tsp] dried oregano
    \item[1 tsp] black pepper
    \item[2 cups] beef broth
    \item[4] zucchini, diced
    \item[1/2 cup] queso anejo or Parmesan, finely grated 
  \end{ingredients}

  \switchcolumn

  \begin{timeline}
    \item[Prep:]  01:00
    \item[Cook:]  00:30
    \item[Total:] 01:30
  \end{timeline}
\end{IT}

\begin{directions}
  \item Roast unpeeled garlic on skillet, turning occasionally until soft, about
    15 minutes.

  \item Broil 3 lbs of the tomatoes on baking sheet \about 4 inches below boiler
    until blackened on one side, about 6 minutes. Flip and broil until other
    side is blackened. Cool and peel.

  \item Brown chorizo in large pot and set aside.

  \item Pour 6 Tbsp of oil into skillet and heat. When quite hot, place nests in
    oil and fry until browned, about 1 minute. Flip and brown the other side.
    Transfer to a paper towel lined tray and repeat (adding additional oil)
    until all nests are browned.

  \item Core and dice the remaining 2 lbs tomatoes. Dice the zucchini.

  \item Process broiled tomatoes (incl.\ collected juices), garlic, and chilies
    until pureed.

  \item Heat 1 Tbsp oil in large pot used for chorizo and add tomato puree. Add
    salt and sear over high heat for 10 minutes. Add the broth, oregano, black
    pepper, and diced tomatoes. Vigorously simmer for 10 minutes.

  \item Add the noodles. After 2 minutes, pull the nests apart. Add the zucchini
    and chorizo and cook an additional 2 minutes. Remove from heat and stir in
    the cheese.
  % \columnbreak
\end{directions}
%---------------------------------------------------------------------------}}}1
\recipe{Thanksgiving} \label{recipe:thanksgiving} % {{{1
\index{Thanksgiving}
\index{turkey!Thanksgiving}

\subsection{Sides}
\begin{twocols} \begin{ingredientslist}
  \item mashed potatoes
  \item sweet potato casserole
  \item green beans or asparagus
  \item biscuits
  \item cranberry relish
  \item pumpkin pie
\end{ingredientslist} \end{twocols}

\subsection{Notes}
\begin{twocols} \begin{kitchennotes}
  \item[Turkey] \begin{kitchensubnotes}
    \item \todo[cf.\ \CICookbook[353] and \CIMeat[439]]
    \item Brine the day before in the morning. Use 1.5 cup of salt. Dry out and
      leave overnight in the refrigerator.
    \item Roast at 275\degF using convection, rotating every hour. \todo[how
      long?]
  \end{kitchensubnotes}

  \item[Gravy] \begin{kitchensubnotes}
    \item 1.5 cups of fat \emph{should} be enough.
    \item Use \todo[how much?] cups of broth or water.
  \end{kitchensubnotes}

  \item[Potatoes]
  Cut potatoes can be left in water for up to 1 day.

  \item[Dressing]
  Corn bread should be cooked day prior in shallow tray.

  \item[Sweet Potatoes]
  Can be roasted the day prior in deep tray.

  \item[Miscellaneous]
    Dressing and sweet potatoes can be cooked at the same time.
\end{kitchennotes} \end{twocols}
%---------------------------------------------------------------------------}}}1
\recipe{Notes} \label{notes:dinners_leftovers} % {{{1
\begin{kitchennotes}
  \item[Pork Roast] \index{Pork Roast} \index{pork!Pork Roast}
    Use pork shoulder or pork butt. Score fat cap and use a rub from
    \namepageref{recipe:rubs_for_roasts}. Roast at 325\degF convection for
    \about 5 hours, until roast registers 190\degF. See also, \CICookbook[418].
  \item[Leg of Lamb] \index{Leg of Lamb} \index{lamb!Leg of Lamb}
    Score fat cap and use a rub from \namepageref{recipe:rubs_for_roasts}. Roast
    at 325\degF convection until done --- 125\degF, about 1.5 hours. Rest of 15
    minutes before carving.
\end{kitchennotes}
%---------------------------------------------------------------------------}}}1

\chaptertoc{Dinners (standalone)} \label{chap:dinners_standalone}
\erecipe{BBQ Chicken}{NA} % {{{1
\index{BBQ Chicken}
\index{chicken!BBQ Chicken}
%---------------------------------------------------------------------------}}}1
\recipe{Chicken Kebabs} \label{recipe:chicken_kebabs} % {{{1
\index{Chicken Kebabs}
\index{chicken!Chicken Kebabs}

\begin{IT}[0.69]
  \begin{ingredients}[Ingredients (yields 4 skewers)]
    \item[1/4 cup] tomato paste
    \item[3 Tbsp] vegetable oil
    \item[2 Tbsp] fish sauce
    \item[3] garlic cloves, grated
    \item[1.5 tsp] sugar
    \item[1.25] kosher salt
    \item[1 Tbsp] hot sauce
    \item[2 lbs] boneless skinless chicken thighs, cut into 1-inch pieces
  \end{ingredients}

  \switchcolumn

  \begin{timeline}
    \item[Prep:]  00:45 + 1:00 marinate
    \item[Cook:]  00:15
    \item[Total:] 2:00
  \end{timeline}
\end{IT}

\begin{directions}
  \item Whisk tomato paste, vegetable oil, fish sauce, garlic, sugar, salt, and
  hot sauce in large bowl. Add chicken to bowl and toss to coat. Cover and
  marinate for 1-12 hours.

  \item Heat grill for 15 minutes. While grill is heating, thread the chicken
  pieces on to the skewers. Lower heat to medium-high, and cook the kebabs,
  turning a quarter turn every 5 minutes. Cook until meat registers 175\degF,
  about 12 minutes.
\end{directions}
%---------------------------------------------------------------------------}}}1
\erecipe{Chicken Thighs}{NA} % {{{1
\index{Chicken Thighs}
\index{chicken!Chicken Thighs}
%---------------------------------------------------------------------------}}}1
\erecipe{Crab Cakes}{\CICookbook[450]} % {{{1
\index{Crab Cakes|see{\CICookbook[450]}}
\index{crab!Crab Cakes|see{\CICookbook[450]}}
%---------------------------------------------------------------------------}}}1
\recipe{Curry Roasted Chicken Drumsticks} \label{recipe:curry_roasted_chicken_drumsticks} % {{{1
\index{Curry Roasted Chicken Drumsticks}
\index{chicken!Curry Roasted Chicken Drumsticks}

\begin{IT}
  \begin{ingredients}
    \item[2.5 lbs] chicken drumsticks
    \item[4 tsp] curry powder
    \item[1 Tbsp] kosher salt
    \item[1.25 tsp] ground cardamom
    \item[1.25 tsp] brown sugar
    \item[3/4 tsp] ground ginger
    \item[3/4 tsp] ground nutmeg
    \item[3/4 tsp] smoked paprika
    \item[2 Tbsp + 1 tsp] warm water
    \item[2 Tbsp + 1 tsp] olive oil
  \end{ingredients}

  \switchcolumn

  \begin{timeline}
    \item[Prep:]  00:45
    \item[Cook:]  1:00
    \item[Total:] 1:45
  \end{timeline}
\end{IT}

\begin{directions}
  \item Adjust oven rack to upper-middle position and heat over to 350\degF.
    Line rimmed baking sheet with aluminum foil and set wire rack in sheet.

  \item Whisk curry powder, salt, cardamom, sugar, ginger, nutmeg, and paprika
    together in small bowl. Add water and oil and stir. Let sit until thickened
    into a paste, 2--3 minutes.

  \item Pat drumsticks dry and brush spice slurry on all sides. Place skin side
    up on wire rack.

  \columnbreak

  \item Roast until drumsticks register 180\degF, 45--50 minutes, rotating
    sheet halfway through. Turn on broiler and continue to cook until drumsticks
    are well-browned in spots, 2--3 minutes longer. Remove from oven and let
    rest for 10 minutes.

  \item Meanwhile, prepare the herbed yogurt sauce below.
\end{directions}

\smallskip

\begin{twocols}
  \begin{ingredients}[Herbed Yogurt Ingredients]
    \item[1 cup] plain Greek yogurt
    \item[1/2 cup] cilantro, minced
    \item[3 Tbsp] lime juice (1--2 limes)
    \item[2 Tbsp] water
    \item[1/2 tsp] fresh ginger, grated
    \item[1/2 tsp] honey
  \end{ingredients}

  \columnbreak

  \begin{directions1col}[Herbed Yogurt Directions]
  \item Whisk all ingredients in a bowl until smooth. If necessary, add up to 2
    Tbsp additional water so that the sauce is pourable. Season with salt to
    taste and refrigerate.
  \end{directions1col}
\end{twocols}
%---------------------------------------------------------------------------}}}1
\erecipe{Fried Chicken}{\CIMeat[404]} % {{{1
\index{Fried Chicken|see{\CIMeat[404]}}
\index{chicken!Fried Chicken|see{\CIMeat[404]}}
%---------------------------------------------------------------------------}}}1
\erecipe{Hot Dogs and Hamburgers}{\textit{see} \namepageref[ ]{notes:dinners_standalone}}
\recipe{Indian Eggs} \label{recipe:indian_eggs} % {{{1
\index{Indian Eggs}
\index{eggs!Indian Eggs}

\begin{IT}
  \begin{ingredients}
    \itemNL oil
    \item[2 tsp] cumin seeds
    \item[2 small] onions
    \itemNL hot peppers
    \item[3 in.\ piece] ginger
    \item[6 cloves] garlic
    \item[2 tsp] turmeric
    \item[1/2 tsp] ground cumin
    \item[2 tsp] chili powder
    \item[1/2 tsp] black pepper
    \item[5] eggs
    \item[2] tomatoes (optional)
    \itemNL red cabbage (optional)
    \item[1/2 tsp] garam masala
    \itemNL salt to taste
  \end{ingredients}

  \switchcolumn

  \begin{timeline}
    \item[Prep:]  \todo
    \item[Cook:]  \todo
    \item[Total:] \todo
  \end{timeline}
\end{IT}

\begin{directions}
  \item Heat oil in a pot over medium high heat until shimmering. Add the cumin
    seeds and saute for 2 minutes until the seeds change color, stirring to
    prevent the spices from burning.
  \item Reduce heat to medium and add the onion, sauteing until the onion begins
    to turn light brown. Add the hot peppers, ginger, garlic, turmeric, ground
    cumin, chili powder, and black pepper, stirring frequently to prevent
    burning, and cook 2 to 3 minutes until quite fragrant.
  % \columnbreak
  \item Add the eggs and continue cooking until done. Off heat, add garam masala
    and salt.
\end{directions}

\todo[When cooking eggs, try covering the pan.]

\todo[Tomatoes and red cabbage?]
%---------------------------------------------------------------------------}}}1
\erecipe{Pan-Seared Salmon}{\CICookbook[443]} % {{{1
\index{Pan-Seared Salmon|see{\CICookbook[443]}}
\index{fish!Pan-Seared Salmon|see{\CICookbook[443]}}
%---------------------------------------------------------------------------}}}1
\erecipe{Pan-Seared Shrimp}{\CICookbook[455]} % {{{1
\index{Pan-Seared Shrimp|see{\CICookbook[455]}}
\index{shrimp!Pan-Seared Shrimp|see{\CICookbook[455]}}
%---------------------------------------------------------------------------}}}1
\erecipe{Red Sauce (for pasta)}{\ATKThirty[157]} % {{{1
\index{Red Sauce (for pasta)|see{\ATKThirty[157]}}
\index{pasta!Red Sauce (for pasta)|see{\ATKThirty[157]}}
%---------------------------------------------------------------------------}}}1
\erecipe{Shrimp Scampi}{\CICookbook[455]} % {{{1
\index{Shrimp Scampi|see{\CICookbook[455]}}
\index{shrimp!Shrimp Scampi|see{\CICookbook[455]}}
%---------------------------------------------------------------------------}}}1
\erecipe{Steak}{\textit{see} \namepageref[ ]{notes:dinners_standalone}}
\erecipe{Tacos}{\textit{see} \namepageref[ ]{notes:dinners_standalone}}
\recipe{Vietnamese Braised Chicken} \label{recipe:vietnamese_braised_chicken} % {{{1
\index{Vietnamese Braised Chicken}
\index{chicken!Vietnamese Braised Chicken}

\begin{IT}
  \begin{ingredients}
    \item[2 lbs] chicken thighs, bone-in, skin-on
    \item[3 inches] ginger, cut into matchsticks
    \item[1] shallot, sliced thin
    \item[2] garlic cloves, sliced thin
    \itemNL hot peppers, to taste (e.g.\ 8), sliced thin
    \item[4 tsp] sugar
    \item[2 tsp] vegetable oil
    \item[1/2 cups] coconut milk
    \item[1 cup] chicken broth
    \item[(or) 1.5 cups] coconut water (omit coconut milk and broth)
    \item[2 Tbsp] fish sauce
    \itemNL cilantro, coarsely chopped
  \end{ingredients}

  \switchcolumn

  \begin{timeline}
    \item[Prep:]  00:30
    \item[Cook:]  01:00
    \item[Total:] 01:30
  \end{timeline}
\end{IT}

\begin{directions}
  \item Place chicken thighs skin-down on cutting board. Trim excess fat and
    skin and de-bone. Keeping as much skin as possible, cut into 1.5 inch
    pieces.

  \item Scatter sugar evenly in bottom of medium saucepan. Add oil and do not
    stir. Cook on medium-high heat without stirring until sugar becomes the
    color of honey, 2--3 minutes. Reduce heat to medium-low, and cook until
    bubbles form and sugar is the color of soy sauce, stirring frequently, about
    1 minute.

  \item Immediately add the ginger, shallot, and garlic and increase the heat to
    medium-high. Cook, stirring frequently, until aromatic, about 30 seconds.
    Add the chicken and cook, stirring frequently, until no longer pink. Add the
    sliced peppers.

  \columnbreak

  \item Add the coconut milk broth mixture or coconut water and the fish sauce.
    Bring to a boil, then reduce heat and simmer until reduced by half, 25--30
    minutes. At this point, begin cooking the rice and prepping any other sides.

  \item Off heat add chopped cilantro and soy sauce to taste.
\end{directions}
%---------------------------------------------------------------------------}}}1
\erecipe{Vodka Cream Sauce (for pasta)}{\ATKThirty[160]} % {{{1
\index{Vodka Cream Sauce (for pasta)|see{\ATKThirty[160]}}
\index{pasta!Vodka Cream Sauce (for pasta)|see{\ATKThirty[160]}}
%---------------------------------------------------------------------------}}}1
\recipe{Notes} \label{notes:dinners_standalone} % {{{1
\begin{kitchennotes}
  \item[Hot Dogs and Hamburgers] \index{Hot Dogs and Hamburgers} \index{beef!Hamburgers}
    For 1 lb of meat, use 1 packet of Lipton's French Onion Soup mix and 1 Tbsp
    of Worcestershire sauce.
  \item[Steak] \index{Steak} \index{beef!Steak}
    Cook on high heat, turning every minute, for 5--6 minutes.
  \item[Tacos] \index{Tacos} \index{beef!Tacos}
    Get a seasoning with cornstarch or flour in it. 1 lb ground beef is enough
    for 2-3 people.
\end{kitchennotes}
%---------------------------------------------------------------------------}}}1

\chaptertoc{Sides} \label{chap:sides}
\recipe{Apple Salad} \label{recipe:apple_salad} % {{{1
\index{Apple Salad}
\index{salad!Apple Salad}

\begin{IT}
  \begin{ingredients}
  \item[6] apples, 3--4 granny smith, the rest sweet
  \item[1/2 block]  cheddar cheese, sharp
  \item[1 cup] raisins or craisins
  \item[1 cup] walnuts or pecans, chopped
  \item[4 stalks] celery
  \item[3/4 large container] plain yogurt \todo[actually, how much?]
  \item[2 Tbsp] brown sugar
  \item[1/2] lemon, juiced
  \end{ingredients}

  \switchcolumn

  \begin{timeline}
    \item[Prep:]  00:30
    \item[Cook:]  N/A
    \item[Total:] 00:30
  \end{timeline}
\end{IT}

\begin{directions}
  \item Cut all ingredients into bite-sized pieces. Leave apple skin on if
    desired.
  \item Prepare the dressing. Mix brown sugar with lemon juice until fully
    dissolved, then add to yogurt. Taste using an apple cube and add additional
    lemon or sugar if needed.

  \columnbreak

  \item Combine all ingredients.
\end{directions}
%---------------------------------------------------------------------------}}}1
\recipe{Asian Pasta Salad} \label{recipe:asian_pasta_salad} % {{{1
\index{Asian Pasta Salad}
\index{pasta!Asian Pasta Salad} \index{salad!Asian Pasta Salad}

\begin{IT}
  \begin{ingredients}
    \item[8 oz] pasta
    \item[8 oz] frozen edamame
    \item[11 oz] green coleslaw mix
    \item[4] green onions, white parts chopped, green parts sliced thin
    \item[1/3 cup] chopped ginger
    \item[1/3 cup] rice vinegar
    \item[1.5 Tbsp] hot sauce
    \item[2 tsp] toasted sesame oil
    \item[1.25 tsp] salt
    \item[1/2 cup] vegetable oil
    \item[2 tsp] toasted black sesame seeds
  \end{ingredients}

  \switchcolumn

  \begin{timeline}
    \item[Prep:]  00:30
    \item[Cook:]  00:15
    \item[Total:] 00:45
  \end{timeline}
\end{IT}

\begin{directions}
  \item Boil water and add pasta, cooking until al dente. Add edamame a cook 2
  minutes more. Drain and rinse with cold water. Add coleslaw mix and the green
  parts of the green onion, stirring to combine.

  \item Add green onion whites, ginger, vinegar, hot sauce, sesame oil, and salt
  to blender and blend until coarse paste forms, \about 30 seconds. Add oil
  and blend until dressing is emulsified and smooth, \about 1 minute.

  \item Toss dressing with salad. Sprinkle sesame seeds in while tossing.
\end{directions}
%---------------------------------------------------------------------------}}}1
\recipe{Classic Pasta Salad} \label{recipe:classic_pasta_salad} % {{{1
\index{Classic Pasta Salad}
\index{pasta!Classic Pasta Salad} \index{salad!Classic Pasta Salad}

\begin{IT}
  \begin{ingredients}
    \item[8 oz] pasta
    \item[3/4 tsp] salt
    \item[1 lb] broccoli crowns, cut into bite-size pieces
    \item[1 can] cannellini beans
    \item[1/4 cup + 1/4 cup] jarred banana peppers
    \item[1/4 cup] banana pepper brine
    \item[1/4 cup + 1/4 cup] Castelvetrano olives, pitted, halved
    \item[2] garlic cloves
    \item[1/2 cup] olive oil
    \item[1/2 cup] olive oil
    \item[10 oz] cherry tomatoes, halved
  \end{ingredients}

  \switchcolumn

  \begin{timeline}
    \item[Prep:]  00:45
    \item[Cook:]  00:00
    \item[Total:] 00:45
  \end{timeline}
\end{IT}

\begin{directions}
  \item Line baking sheet with paper towels. Boil water and add pasta, cooking
  until tender, about 5 minutes. Add broccoli and beans and cook until tender,
  1-2 minutes. Drain and spread on prepared baking sheet.

  \item Combine 1/4 cup peppers, brine, 1/4 cup olives, garlic, and salt in
  blender until coarse paste forms, \about 30 seconds. Add oil and continue to
  blend until dressing is emulsified and smooth.

  % \columnbreak

  \item Combine pasta mixture, tomatoes, 1/4 cup banana peppers, 1/4 cup olives
  and dressing. Toss to coat.
\end{directions}
%---------------------------------------------------------------------------}}}1
\erecipe{Coleslaw}{\CICookbook[57]} % {{{1
\index{Coleslaw|see{\CICookbook[57]}}
%---------------------------------------------------------------------------}}}1
\recipe{Cranberry Relish} \label{recipe:cranberry_relish} % {{{1
\index{Cranberry Relish}

\begin{IT}[0.770]
  \begin{ingredients}
    \item[1/2 cup] sugar
    \item[1] clementine, unpeeled, stemmed and quartered
    \item[12 oz (3 cups)] frozen cranberries, thawed
  \end{ingredients}

  \switchcolumn

  \begin{timeline}
    \item[Prep:]  00:10
    \item[Cook:]  1:00 (chilling)
    \item[Total:] 1:10
  \end{timeline}
\end{IT}

\begin{directions}
  \item Process sugar and clementine in food processor until clementine is finely
    ground, scraping the bowl as needed, about 20 seconds.

  \item Add cranberries and pulse until berries are chopped into pea-sized
    pieces, about 15-20 pulses.

  \columnbreak

  \item Transfer to bowl and refrigerate for at least 1 hour, or up to 24 hours.
    Stir to recombine.
\end{directions}
%---------------------------------------------------------------------------}}}1
\erecipe{Cucumber Salad}{\CICookbook[45]} % {{{1
\index{Cucumber Salad|see{\CICookbook[45]}}
\index{salad!Cucumber Salad|see{\CICookbook[45]}}
%---------------------------------------------------------------------------}}}1
\erecipe{Hummus}{\CICookbook[8]} % {{{1
\index{Hummus|see{\CICookbook[8]}}
%---------------------------------------------------------------------------}}}1
%\recipe{Indian Carrots, Potatoes, Peas} \label{recipe:indian_carrots,_potatoes,_peas} % {{{1
%\index{Indian Carrots, Potatoes, Peas}
%\index{carrots!Indian Carrots, Potatoes, Peas}

%\begin{IT}
%  \begin{ingredients}
%    \item \vspace{15em}
%  \end{ingredients}

%  \switchcolumn

%  \begin{timeline}
%    \item[Prep:]
%    \item[Cook:]
%    \item[Total:]
%  \end{timeline}
%\end{IT}

%\begin{directions}
%  \item 
%  % \columnbreak
%\end{directions}
%%---------------------------------------------------------------------------}}}1
%\recipe{Indian Cauliflower} \label{recipe:indian_cauliflower} % {{{1
%\index{Indian Cauliflower}
%\index{cauliflower!Indian Cauliflower}

%\begin{IT}
%  \begin{ingredients}
%  \item \vspace{15em}
%  \end{ingredients}

%  \switchcolumn

%  \begin{timeline}
%    \item[Prep:]
%    \item[Cook:]
%    \item[Total:]
%  \end{timeline}
%\end{IT}

%\begin{directions}
%  \item
%  % \columnbreak
%\end{directions}
%%---------------------------------------------------------------------------}}}1
\recipe{Mashed Sweet Potatoes} \label{recipe:mashed_sweet_potatoes} % {{{1
\index{Mashed Sweet Potatoes}
\index{sweet potatoes!Mashed Sweet Potatoes}

\begin{IT}
  \begin{ingredients}
    \item[2 lbs] sweet potatoes, peeled and cut into 1/2-inch pieces
    \item[3 Tbsp] butter, cut into 1/2-inch pieces
    \item[1/4 tsp] salt
  \end{ingredients}

  \switchcolumn

  \begin{timeline}
    \item[Prep:]  00:05
    \item[Cook:]  00:25
    \item[Total:] 00:30
  \end{timeline}
\end{IT}

\begin{directions}
  \item Bring 6 cups of water to boil in large saucepan over high heat. Add
    sweet potatoes, return to a boil, cover, reduce heat to medium-low to
    simmer, and cook until fork pierces potatoes with little resistance, about
    12 minutes.

  \item Drain potatoes and cook over medium heat, stirring frequently, for 7
    minutes, adjusting heat to prevent excessive browning (steam should be
    steadily escaping).

  \columnbreak

  \item Off heat, add butter and salt. Whisk until butter is fully incorporated.
    Season with salt to taste.
\end{directions}
%---------------------------------------------------------------------------}}}1
\erecipe{Indian Green Beans with Mustard}{\InvIndianCooking[152]} % {{{1
\index{Indian Green Beans with Mustard|see{\InvIndianCooking[152]}}
\index{green beans!Indian Green Beans with Mustard|see{\InvIndianCooking[152]}}
%---------------------------------------------------------------------------}}}1
\erecipe{Potato Salad}{\CICookbook[59]} % {{{1
\index{Potato Salad|see{\CICookbook[59]}}
\index{salad!Potato Salad|see{\CICookbook[59]}}
%---------------------------------------------------------------------------}}}1
\erecipe{Rice Pilaf}{\CICookbook[222]} % {{{1
\index{Rice Pilaf|see{\CICookbook[222]}}
\index{rice!Rice Pilaf|see{\CICookbook[222]}}
%---------------------------------------------------------------------------}}}1

\chaptertoc{Breakfast} \label{chap:breakfast}
\erecipe{French Toast}{\CICookbook[551]} % {{{1
\index{French Toast|see{\CICookbook[551]}}
%---------------------------------------------------------------------------}}}1
\erecipe{Lemon Blueberry Pancakes}{\CIBaking[87]} % {{{1
\index{Lemon Blueberry Pancakes|see{\CIBaking[87]}}
\index{pancakes!Lemon Blueberry Pancakes|see{\CIBaking[87]}}
%---------------------------------------------------------------------------}}}1

\chaptertoc{Baked Goods} \label{chap:baked_goods}
\erecipe{Apple Pie}{\CICookbook[713]} % {{{1
\index{Apple Pie|see{\CICookbook[713]}}
\index{pie!Apple Pie|see{\CICookbook[713]}}
%---------------------------------------------------------------------------}}}1
\erecipe{Angel Food Cake}{\CIBaking[270]} % {{{1
\index{Angel Food Cake|see{\CIBaking[270]}}
\index{cake!Angel Food Cake|see{\CIBaking[270]}}
%---------------------------------------------------------------------------}}}1
\recipe{Baking Powder Biscuits} \label{recipe:baking_powder_biscuits} % {{{1
\index{Baking Powder Biscuits}
\index{biscuits!Baking Powder Biscuits}

\begin{IT}
  \begin{ingredients}[Ingredients (yields 4 biscuits)]
    \item[180 g] flour
    \item[1/2 tsp] salt
    \item[1/2 Tbsp] baking powder
    \item[1/2 Tbsp] sugar
    \item[3 Tbsp] butter, \textbf{room temperature}
    \item[118 g] milk
  \end{ingredients}

  \switchcolumn

  \begin{timeline}
    \item[Prep:]  00:20
    \item[Cook:]  00:20
    \item[Total:] 00:40
  \end{timeline}
\end{IT}

\begin{directions}
  \item Preheat oven to 425\degF.

  \item Mix flour, salt, baking powder, and sugar in food processor.

  \item Add butter to processor and process until mixture has a sandy
    consistency.

  \item Slowly add milk to running processor until a cohesive dough forms. If
    dough seems dry, add additional milk.

  \columnbreak

  \item Roll out the dough on a floured work surface with floured rolling pin,
    folding and rolling repeatedly. Form dough into 3/4 inch disc.

  \item Cut dough into biscuits and place biscuits onto baking sheet. Brush tops
    of biscuits with milk, and place in oven for 15--20 minutes until lightly
    browned.
\end{directions}
%---------------------------------------------------------------------------}}}1
\erecipe{Banana Bread}{\CIBaking[27]} % {{{1
\index{Banana Bread|see{\CIBaking[27]}}
\index{bread!Banana Bread|see{\CIBaking[27]}}
%---------------------------------------------------------------------------}}}1
\recipe{Banana Chocolate Chip Muffins} \label{recipe:banana_chocolate_chip_muffins} % {{{1
\index{Banana Chocolate Chip Muffins}
\index{muffins!Banana Chocolate Chip Muffins}

\begin{IT}
  \begin{ingredients}
    \item[8 Tbsp] butter, room temperature
    \item[87 g] sugar
    \item[2] ripe bananas
    \item[1] egg
    \item[1 tsp] vanilla
    \item[1/3 cup] milk
    \item[1 cup] all purpose flour
    \item[1.5 tsp] baking powder
    \item[1/2 tsp] baking soda
    \item[1/2 tsp] cinnamon
    \item[1/2 tsp] salt
    \item[1/2 cup] bittersweet chocolate chips
    \item[1/2 cup] cacao nibs
  \end{ingredients}

  \switchcolumn

  \begin{timeline}
    \item[Prep:]  00:10
    \item[Cook:]  00:20
    \item[Total:] 00:30
  \end{timeline}
\end{IT}

\begin{directions}
  \item Preheat oven to 350\degF.
  \item Beat together butter and sugar until smooth.
  \item Beat in the bananas, egg, vanilla, and milk.
  \item Add flour, baking powder, baking soda, cinnamon, and salt and beat until
    smooth.

  \columnbreak

  \item Add chocolate chips and cacao nibs, stirring until combined.
  \item Heap batter into muffin tray and bake for 20--24 minutes, rotating
    halfway.
\end{directions}
%---------------------------------------------------------------------------}}}1
\erecipe{Blueberry Muffins}{\CIBaking[42]} % {{{1
\index{Blueberry Muffins|see{\CIBaking[42]}}
\index{muffins!Blueberry Muffins|see{\CIBaking[42]}}
%---------------------------------------------------------------------------}}}1
\erecipe{Carrot Cake}{\CIBaking[313]} % {{{1
\index{Carrot Cake|see{\CIBaking[313]}}
\index{cake!Carrot Cake|see{\CIBaking[313]}}
%---------------------------------------------------------------------------}}}1
\erecipe{Chocolate Layer Cake}{\CIBaking[299]} % {{{1
\index{Chocolate Layer Cake|see{\CIBaking[299]}}
\index{cake!Chocolate Layer Cake|see{\CIBaking[299]}}
%---------------------------------------------------------------------------}}}1
\recipe{French Country Bread} \label{recipe:french_country_bread} % {{{1
\index{French Country Bread}
\index{bread!French Country Bread}

\begin{timeline}
  \item[Prep:]  00:05 (biga) + 00:30 (bread)
  \item[Cook:]  00:40
  \item[Total:] (2:00--16:00 biga) + (00:30 prep) + (1:15--3:30 rise/proof)
    + (00:40 cook) = 5:25--20:40
\end{timeline}

\smallskip

\begin{twocols}
  \begin{ingredients}[Biga Ingredients]
    \item[150 g] bread flour
    \item[30 g] whole wheat flour
    \item[1/2 tsp] instant yeast \textbf{or} active dry yeast
    \item[1 cup] water, 90--100\degF
  \end{ingredients}

  \columnbreak

  \begin{ingredients}[Bread Ingredients]
    \item[533 g] bread flour
    \item[18 g] sugar
    \item[1 Tbsp] salt
    \item[2/3 tsp] instant yeast or \textbf{1 tsp} active dry yeast
    \item[1 cup] water, 105--115\degF
  \end{ingredients}
\end{twocols}

\begin{directions}
  \item Combine all biga ingredients, proofing the yeast in the water before
  adding it. Cover with plastic wrap and let rest 2--16 hours (longer is
  better).

  \item Stir down the biga and combine with the bread ingredients, once again
  proofing the yeast in the water before adding it. Stir until roughly
  incorporated, then let rest 15 minutes.

  \item After resting, knead the dough, adding enough flour for it to form a
  soft dough. Knead for 10 minutes.

  \item Let the dough rise in a lightly greased covered container until doubled
  in size, 45 minutes to 2 hours.

  \columnbreak

  \item Remove the dough from the container and gently form it into a round by
  flouring the counter and shaping it. Transfer to a lightly floured or
  cornmealed cloche or baking sheet. Cover and proof until 50\% larger, 30--90
  minutes. Preheat oven to 450\degF.

  \item Score the top of the bread and dust with flour. Bake at 450\degF for 35
  minutes in a covered cloche, then remove the cover and bake for an additional
  5. If using a baking sheet then bake for 40 minutes. Remove from oven and let
  cool on a rack.
\end{directions}
%----------------------------------------------------------------------------}}}1
\erecipe{Gingerbread Cake}{\CIBaking[264]} % {{{1
\index{Gingerbread Cake|see{\CIBaking[264]}}
\index{cake!Gingerbread Cake|see{\CIBaking[264]}}
%---------------------------------------------------------------------------}}}1
\erecipe{Gingerbread Cookies}{\CICookbook[642]} % {{{1
\index{Gingerbread Cookies|see{\CICookbook[642]}}
\index{cookies!Gingerbread Cookies|see{\CICookbook[642]}}
%---------------------------------------------------------------------------}}}1
\erecipe{Pumpkin Pie}{\CICookbook[720]} % {{{1
\index{Pumpkin Pie|see{\CICookbook[720]}}
\index{pie!Pumpkin Pie|see{\CICookbook[720]}}
%---------------------------------------------------------------------------}}}1
\recipe{Savory Dutch Baby} \label{recipe:savory_dutch_baby} % {{{1
\index{Savory Dutch Baby}
\index{dutch baby, savory}
\begin{IT}[0.727]
  \begin{ingredients}[Dutch Baby Ingredients]
    \item[1.75 cups (8.76 oz)] all-purpose flour
    \item[1 Tbsp] sugar
    \item[1/2 tsp] salt
    \item[1.5 cups] milk
    \item[6] eggs
    \item[3 Tbsp] butter
  \end{ingredients}

  \switchcolumn

  \begin{timeline}
    \item[Prep:]  00:20 (dutch baby) + 00:20 (topping)
    \item[Cook:]  00:35
    \item[Total:] 00:55
  \end{timeline}
\end{IT}

\begin{directions}[Dutch Baby Directions]
  \item Whisk flour, sugar, and salt together in large bowl. Whisk milk and eggs
    together in a second bowl. Whisk 2/3 of milk mixture into flour mixture
    until no clumps remain, then slowly whisk in remaining milk mixture until
    smooth.

  \item Adjust oven rack to lower-middle position. Melt butter in 12-inch
    nonstick skillet (or use cooking oil spray in traditional skillet) over
    medium-low heat. Add batter and transfer to cold oven. Set oven to 375\degF
    and bake until edges are deep golden brown and center is beginning to brown,
    30--35 minutes.

  \item While baking, prepare the topping (see below).

  \item Gently transfer to cutting board and let cool for at least 5 minutes
    before topping.
\end{directions}

\smallskip

\begin{twocols}
  \begin{ingredients}[Lox and Cucumber Ingredients]
    \item[1] shallot, small
    \item[1/4 tsp + 1/4 tsp] salt, divided
    \item[1/2 cup] sour cream
    \item[1 Tbsp] capers, plus \textbf{1 Tbsp} brine
    \item[1.5 tsp + 1.5 tsp] fresh dill, chopped, divided
    \item[1] English cucumber
    \item[8 oz] smoked salmon
  \end{ingredients}

  \begin{directions1col}[Lox and Cucumber Directions]
    \item Combine 1/2 cup water, shallot, and 1/4 tsp salt in a bowl and let sit
      for 5 minutes. Drain and discard liquid, setting shallot aside. Combine sour
      cream, caper brine, 1.5 tsp dill, and 1/4 tsp salt in a bowl.
    \item Shave cucumber using vegetable peeler lengthwise into ribbons until
      hitting seeds. Rotate 180\textdegree\ and repeat. Discard core.
    \item Spread sour cream mixture over Dutch Baby. Arrange salmon, then scatter
      cucumber ribbons, then sprinkle with shallot, capers, and 1.5 tsp dill.
  \end{directions1col}

  \columnbreak

  \begin{ingredients}[Mushroom and Red Pepper Ingredients]
    \item[2 Tbsp + 1 Tbsp] olive oil, divided
    \item[1.25 lbs] Portobello mushroom caps, gills removed, sliced thin
    \item[1/2 tsp + 1/4 tsp] salt, divided
    \item[1/2 tsp] lemon zest plus \textbf{2 tsp} lemon juice
    \item[1/4 tsp] sugar
    \item[1/2 cup] chopped roasted red peppers
    \item[2 oz] feta cheese, cut into 1/4 inch cubes
    \item[1/2 cup] parsley
    \item[1/4 cup] walnuts, chopped
  \end{ingredients}

  \begin{directions1col}[Mushroom and Red Pepper Directions]
    \item Heat 1 Tbsp oil in 12-inch pan over medium-high heat. Add mushrooms
      and 1/2 tsp salt and cook, stirring frequently, until tender, 4--6
      minutes.
    \item Whisk lemon zest, juice, sugar, and 1/4 tsp salt. Whisking constantly,
      slowly add 2 Tbsp olive oil. Add red peppers, feta, and parsley and
      combine.
    \item Spread mushrooms over Dutch Baby, followed by red pepper mixture and
      walnuts.
  \end{directions1col}
\end{twocols}
%---------------------------------------------------------------------------}}}1
\erecipe{Tiramis\`u}{\CICookbook[700]} % {{{1
\index{Tiramis\`u|see{\CICookbook[700]}}
\index{cake!Tiramis\`u|see{\CICookbook[700]}}
%---------------------------------------------------------------------------}}}1
\erecipe{Tripe-Chocolate Mousse Cake}{\CIBaking[322]} % {{{1
\index{Tripe-Chocolate Mousse Cake|see{\CIBaking[322]}}
\index{cake!Tripe-Chocolate Mousse Cake|see{\CIBaking[322]}}
%---------------------------------------------------------------------------}}}1
\erecipe{Ultimate Chocolate Chip Cookies}{\CIScience[415]} % {{{1
\index{Ultimate Chocolate Chip Cookies|see{\CIScience[415]}}
\index{cookies!Ultimate Chocolate Chip Cookies|see{\CIScience[415]}}
%---------------------------------------------------------------------------}}}1

\chaptertoc{Miscellaneous} \label{chap:miscellaneous}
\recipe{Blackened Seasoning} \label{recipe:blackened_seasoning} % {{{1
\index{Blackened Seasoning}
\index{fish!Blackened Seasoning for}

\begin{ingredients}
  \item[2 Tbsp] paprika
  \item[1 Tbsp] cayenne
  \item[1 Tbsp] onion powder
  \item[1 tsp] garlic powder
  \item[1 tsp] black pepper
  \item[1 tsp] salt
  \item[1/2 tsp] dried basil
  \item[1/2 tsp] dried oregano
  \item[1/2 tsp] dried thyme
\end{ingredients}

\begin{directions}
  \item Mix all ingredients.
\end{directions}
%---------------------------------------------------------------------------}}}1
\recipe{Pan Sauces} \label{recipe:pan_sauces} % {{{1
\index{sauces, pan}

\begin{directions}[General Directions]
  \item After saut\'eing meat, tent loosely with foil. Add oil or butter to the
    pan if necessary and saut\'e the aromatics, no more than a couple minutes
    so as not to burn the fond, making it bitter.

  \item Add flour, if using, stirring constantly, for 30 seconds.

  \item Add final aromatics. Deglaze the pan with the liquids. If using wine,
    add wine first and reduce, then add other liquids and reduce. Return juices
    from tented meat to the pan.

  \item Off heat, add acids and butter. Season with salt and pepper to taste.
\end{directions}

\noindent\textbf{Ingredient lists yield two large servings.}

\begin{twocols}
  \begin{ingredients}[Garlic Lemon Thyme]
    \index{sauces, pan!Garlic Lemon Thyme}
    \item[1] shallot, minced
    \item[2] cloves garlic
    \itemNL fresh thyme
    \item[1 cup] chicken broth
    \item[1/2] lemon, juiced
    \item[1 Tbsp] butter
  \end{ingredients}

  \begin{ingredients}[Lemon Caper White Wine]
    \index{sauces, pan!Lemon Caper White Wine}
    \item[1] shallot, sliced thin
    \item[1.5 tsp] capers, chopped
    \item[1/2 cup] white wine or vermouth
    \item[1/2 cup] chicken broth
    \item[1/2] lemon, juiced
    \item[1 Tbsp] butter
  \end{ingredients}

  \begin{ingredients}[Thyme Wine]
  \index{sauces, pan!Thyme Wine}
    \item[1] shallot, minced
    \itemNL fresh thyme
    \item[1/2 cup] wine (white \textbf{or} red)
    \item[1/2 cup] chicken stock
    \item[1 Tbsp] light brown sugar
    \item[1 Tbsp] butter
  \end{ingredients}

  \columnbreak

  \begin{ingredients}[Red Wine Rosemary]
    \index{sauces, pan!Red Wine Rosemary}
    \item[1] shallot, minced
    \itemNL fresh rosemary, chopped
    \item[1/2 cup] red wine
    \item[1/2 cup] chicken \textbf{or} beef broth
    \item[1 Tbsp] butter
  \end{ingredients}

  \begin{ingredients}[Red Wine Mushroom Thyme]
    \index{sauces, pan!Red Wine Mushroom Thyme}
    \item[4 oz] mushrooms, sliced thin (cook first, \about 5 min.)
    \item[1/2] shallot, minced
    \itemNL fresh thyme
    \item[3/2 cup] red wine
    \item[1/4 cup] chicken \textbf{or} beef broth
    \item[1.5 tsp] balsamic vinegar
    \item[1/2 tsp] Dijon mustard
    \item[1 Tbsp] butter
  \end{ingredients}

  % \begin{ingredients}[<++>]
  % \index{sauces, pan!}
  %   \item[<++>] <++>
  %   \item[<++>] <++>
  % \end{ingredients}
\end{twocols}
%---------------------------------------------------------------------------}}}1
\erecipe{Pesto}{\CICookbook[161]} % {{{1
\index{Pesto|see{\CICookbook[161]}}
\index{pasta!Pesto|see{\CICookbook[161]}}
%---------------------------------------------------------------------------}}}1
\recipe{Rubs for Roasts} \label{recipe:rubs_for_roasts} % {{{1
\index{roasts, rubs for}

\begin{twocols}
  \begin{ingredients}[Garlic Rosemary Thyme Dijon]
    \index{roasts, rubs for!Garlic, Rosemary, Thyme, Dijon}
    \item[8] cloves garlic, minced
    \item[2 Tbsp] olive oil
    \item[2 Tbsp] fresh rosemary, chopped
    \item[2 Tbsp] fresh thyme, chopped
    \item[2 Tbsp] Dijon mustard
    \item[1 Tbsp] kosher salt
    \item[2 tsp]  black pepper
  \end{ingredients}

  \columnbreak

  \begin{ingredients}[Roasted Garlic Parsley]
    \index{roasts, rubs for!Roasted Garlic Parsley}
    \item[2 heads] garlic, roasted
    \item[2 Tbsp] parsley, minced
    \itemNL salt and pepper (on roast, after rub)
  \end{ingredients}

  \begin{ingredients}[Salt and Sugar]
    \index{roasts, rubs for!Salt and Sugar}
    \item[1/3 cup] brown sugar
    \item[1/3 cup] kosher salt
  \end{ingredients}
  \vspace{-0.75em} Rub, then cover the roast with plastic wrap and refrigerate
  for 12--24 hours.

  % \begin{ingredients}[<++>]
  %   \index{roasts, rubs for!<++>}
  %   \item[<++>] <++>
  %   \item[<++>] <++>
  % \end{ingredients}
\end{twocols}
%---------------------------------------------------------------------------}}}1

\backmatter
\def\thisrecipe{}  % clear recipe name
\pagenumbering[I]{bychapter} \printindex
\end{document}
